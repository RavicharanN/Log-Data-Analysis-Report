% !TEX root = thesis.tex

\section{Comparative Study}\label{sec:contents}
We implemented multiple approaches to implement the clustering of the log data and are tested for better performances in terms of accuracy and efficiency. The methods implemented include K means clustering, Minhash LSH (tested with shingles of variable sizes), and a Graph based approach that treats all the messages as nodes and the weight of the edges as the distance between the messages. 

\subsection{K means clustering}\label{sec:_before_you_start}


\subsection{Minhash}\label{sec:_the_abstract}
% If your research is going to be sold, the abstract serves as the \textit{teaser}.
% This means that people must be able to decide on the basis of the abstract whether the rest of your article is worth reading.
% In many cases, abstracts are placed in a range of databases and, on the basis of the abstract, readers must decide whether to download or purchase the article – which can be expensive in the latter case.

% Because abstracts relate to academic/ scientific research, it is important that they include the key findings and conclusions.
% An abstract must not be a cliff-hanger that leaves the reader in suspense as to the outcome of the research.
% It must give a complete idea of what happened and what can be deduced from this.

% The abstract must be seen as separate from the rest of the text.
% It should therefore be possible to read the main text without needing to read the abstract first.

\subsection{Wieghted Minhash}\label{sec:_the_abstract}
% If your research is going to be sold, the abstract serves as the \textit{teaser}.
% This means that people must be able to decide on the basis of the abstract whether the rest of your article is worth reading.
% In many cases, abstracts are placed in a range of databases and, on the basis of the abstract, readers must decide whether to download or purchase the article – which can be expensive in the latter case.

% Because abstracts relate to academic/ scientific research, it is important that they include the key findings and conclusions.
% An abstract must not be a cliff-hanger that leaves the reader in suspense as to the outcome of the research.
% It must give a complete idea of what happened and what can be deduced from this.

% The abstract must be seen as separate from the rest of the text.
% It should therefore be possible to read the main text without needing to read the abstract first.

\subsection{Performance with the use of shingles}\label{sec:_subdivide_text}
% A good structure with sections or chapters is important for the legibility of your thesis.
% It is important to think in advance about how you want to structure the material you are going to present.
% There is often a fixed structure (e.g., introduction, system description, results, conclusion, for a AI report; introduction, method, results, discussion, for empirical work) for short texts such as articles and the Bachelor's thesis.
% In the case of longer texts, these parts may be divided into more chapters, but it is also useful to consider whether subsections are required in short texts.
% When writing a PhD thesis, for example, it is even customary to pause half way through the process in order to draw up a detailed table of contents, setting out the chapters, the sections in the chapters and a brief description of the contents.
% This may even be useful for a Bachelor's thesis, and certainly for a Master's thesis.

% Obviously, the content of each subdivision must form a cohesive whole.
% Devote each subdivision to a specific subject, and decide in advance what should and should not be included.
% Sooner or later you may encounter the problem that you need to have explained point B before explaining point A, and vice versa.
% You can solve this by referring ahead, and possibly giving a brief idea of what point B involves.
% A little redundancy in your thesis is therefore acceptable.
% Avoid endlessly long sections in which you try to explain everything at once.

% Ideally, subsections, sections and chapters will all be roughly the same length.
% Although differences in length are obviously inevitable, there is nevertheless a problem with the structure if you have sections with only one paragraph as well as sections that are three pages long.
% The use of many levels of  subheadings (e.g.\ 1.2.1.7b) also points to a problem with the structure.
% A rule of thumb for longer texts is that it must be possible to read a chapter in one sitting (1½ hours or so).
% A good length is therefore between 10 and 20 pages.


\subsection{Graph Based clustering methods}\label{sec:_system_methods}
% The description of the system and/or method must focus on reproducibility.
% In the case of a computer program, sufficient detail must be provided to enable the programming to be repeated.
% In the case of an experimental setup, this means that sufficient details should be given so that the precise experiment can be repeated.

% For experimental papers, there is a much used, and therefore more or less standard structure.
% It is  highly recommended to adhere to such a standard structure.
% It helps the reader because they know where to look for particular information, and it helps you as a writer because it is easier not to forget things.

% In both cases, only relevant details should be given.
% Just as, in the case of an fMRI experiment, the colour of the pyjamas worn by the test subjects is unimportant but the type and settings of the MRI machine \textit{are} important, in the case of programs it is important to describe precisely \textit{what} happens, but not precisely \textit{how} this was programmed.
% The description therefore needs to be above the source-code level.
% Often, the best way is to describe in general how the system works and then discuss its individual components.
% Ultimately, everything must be substantiated with formulae and tables of pseudo code.

% The reasons behind certain choices in the system should also be given.
% In many cases, certain aspects of the system can be executed in many different ways.
% It is therefore necessary to explain why a certain method of execution was chosen and why the alternatives were rejected.
% It is helpful to place these choices in the wider context of the research, as outlined in the introduction and research question.
% Describing the system therefore involves more than simply describing what you built!

% Importantly, remember to include the values of all the parameters used in your system! The best way to do this is to present them in a table in a central position.

% The description in the system/method section goes beyond the description of the computer model or experiment that was used.
% The techniques used to analyse the behaviour of the system (or the model, or the experimental subjects) should also be described.
% In many cases, a computer model or an experiment generates sizeable volumes of data that, in their raw form, are too large and complex to draw a conclusion from.
% In such cases, measures need to be defined for calculations using the data.
% These may be averages, or aspects such as the diversity of the system, learning success, rate of convergence, etc.
% The calculations must be clearly described, and reasons given for the choice of measures.
% For experiments, this means that the method section also describes how the raw data were collected, averaged, and processed to make it ready for statistical testing.
% The statistical tests must also be described, together with the reasons to use those.

%  idea to present them in a pie chart, and sometimes it is useful to think up a graph of your own.
% A number of conventions for producing decent graphs are discussed in Section 4.7.
% As with writing, when you are learning to produce graphs it is useful to look critically at a wide range of graphs produced by other people.

% Graphs are not the answer to everything.
% Sometimes it is better to present results in a table.
% This is certainly the case if precise numeric values are important, or in the case of very small numbers of data points.
% Sometimes it is sufficient to discuss the results in the text, certainly if the qualitative form of the results is more important than the precise quantitative values.

% In all cases, you must show whether or not your results are statistically significant.
% In graphs this is done using box plots, reliability intervals or error bars.
% In tables it is useful to give reliability intervals and/or standard deviations, etc.
% When discussing your results you must also clearly explain which statistical tests you used and why they are valid tests to use.

% \subsection{The conclusion and discussion}\label{sec:_conclusion_discussion}
% In the conclusion and discussion you describe the consequences of your findings, obviously in the context of your research question.
% As mentioned above, the introduction, together with the conclusion and discussion, must form a mini-article that can be read on a stand-alone basis.
% Therefore you should somehow include all your key findings in this section.
% Clearly, there is no need to repeat graphs and tables, but it is essential to summarize your findings in words.

% The repetition will occur when you explain what the results mean and how they should be interpreted.
% You should take the reader by the hand, as it were, by explaining  how each result is relevant to the research question and to other work on the subject.
% This discussion may serve as an argumentation for a short and (hopefully) powerful message: the ultimate conclusion.
% Obviously, there may be more than one conclusion that can be drawn from your results.

% In the discussion, the conclusion can be placed in the broader context of the academic discussion to which the presented work contributes.
% This is also the place to make suggestions for future research.

% Possibilities for future research are sometimes (but not usually in a Bachelor's thesis) presented in a separate section.
% In this section, keen young academics at the start of their career have the tendency to suggest how the system can be made even more complex, especially if  their results were disappointing.
% But remember that the reverse may be true: perhaps the system was too complex, and simplification would provide greater insight.
% Suggestions like these should always be made with the original research question in mind, not simply because it is enjoyable to play with complex systems.
% When you've done an experiment, don't simply suggest to use more subjects in a next experiment to increase the power of the experiment.
% Really large effects in the real world should have shown in your experiment even with a relatively small sample.
% Don't confuse significance with relevance!

% \subsection{Acknowledgements and related matters}\label{sec:_acknowledgements}
% Acknowledgements etc.\ are usually only appropriate in texts that are longer than a Bachelor's thesis.
% It is acceptable to include acknowledgements in a Master's thesis, and for books and PhD theses it is standard.
% Parents and other loved ones are often very flattered to be mentioned in acknowledgements.
% It is not necessary to mention the supervisor in acknowledgements in texts below the level of a doctoral thesis.
% The supervisor is mentioned anyway.
% Supervisors may be embarrassed to be mentioned in the acknowledgements of a Master's thesis (since supervision is simply part of their work), but tastes differ and it is therefore a good idea to discuss this with your supervisor.

% Generally speaking, a foreword is not required in a thesis.
% A foreword usually says something about the reason for the work and about how the work was carried out.
% It is best to leave this out of a Bachelor's or Master's thesis unless there is something exceptional to report.

% \subsection{The bibliography}\label{sec:_the_bibliography}
% The format for references is described in Section 4.9.
% Obviously, it is also important to know when references are needed, and to which sources.
% References are always required when you use other people’s ideas or when you use text or illustrations that are not your own.
% When describing a system, references must be included when you are explaining the various choices made.
% Choices of this sort must be based on something – preferably a sound knowledge of what other researchers have already tried.
% Obviously, in your introduction you should include references to similar work, but without interrupting the flow of your account.
% Sometimes useful, but very tiring to read, are the literature overviews that summarize a great deal of other work but do not really relate it to the author’s own work.
% In the conclusion, after all, full reference can be made to similar conclusions drawn by fellow scientists or academics.

% A practical problem that may occur is deciding precisely which source to refer to.
% Should you refer to the person who originally discovered something, or to the textbook that gives a much clearer explanation of what it is about? Probably the simplest solution is to refer to both.
% On balance, it is better to over-reference than to under-reference, but be careful about referring to original work that will be difficult for the reader to find.
% You should be open about the fact that you have not read something, for example by including a reference in the following form: (\citealp{schwartz97}, as cited in \citealp{berrah99}).
% In the reference list you then refer to both \citetext{\citeauthor{schwartz97}, without the cited bit, and \citeauthor{berrah99}}.
